% Dmitry Mikushin, USI Lugano, dmitry.mikushin@usi.ch,
% using portions of original style file by Tom Cashman
%
% IMPORTANT NOTICE:
%
% The USI logo is unique; it is authorized for use only by employees of the
% Università della Svizzera italiana for work-related projects; others can use them
% ONLY with prior authorization (contact: press@usi.ch).
%
% http://www.press.usi.ch/en/corporate-design/corporate-design-stampa.htm
%
% This is an example beamer presentation, which uses Università della Svizzera italiana
% design theme.

\documentclass[aspectratio=169]{beamer}

% Packages
\usetheme{usi}
\usepackage{array}
\usepackage[backend=biber]{biblatex}
\usepackage{caption}
\usepackage{float}
\usepackage{graphicx}
\usepackage{hhline}
\usepackage{listings}
\usepackage{multirow}
\usepackage{pgfplots}
\usepackage{subcaption}
\usepackage{tikz}
\usepackage{xcolor}

% Bibliography source.
\addbibresource{references.bib}

% Colour definitions
\definecolor{mycolor3}{RGB}{170,35,3}

\setlength{\fboxsep}{0.25pt}
\setlength{\fboxrule}{0pt}

\title[Particle Simulations with OpenACC]{\textbf{Evaluating large scale particle simulations with OpenACC}\\[0.5em] Status Update}
\author{Samuel A. Cruz Alegr\'{i}a, Alessandra M. de Felice, Hrishikesh R. Gupta}
\institute{(University of Lugano)}
\date{\today}


\begin{document}
%-------------------------------------------------------------------------------	
\begin{frame}
\titlepage
\end{frame}
%-------------------------------------------------------------------------------

%-------------------------------------------------------------------------------
\begin{frame}[fragile]{Tasks so far}

Our tasks for this week were the following:
%
\begin{itemize}
	\item Develop serial code.
	\item Investigate visualization tools.
	\item Investigate parallelization methods.
\end{itemize}
%

\end{frame}
%-------------------------------------------------------------------------------

%-------------------------------------------------------------------------------
\begin{frame}[fragile]{Serial code}
	The serial code is divided into the following three main sections:
	%
	\begin{enumerate}
		\item Tracing particles (trails or no trails).
		\item Drawing particles.
		\item Updating particle details such as position and velocity.
	\end{enumerate}
	%
	
	For the moment, particle movement doesn't strictly abide to any well-established physics. For instance, particle collisions are not calculated yet.
\end{frame}
%-------------------------------------------------------------------------------

%-------------------------------------------------------------------------------
\begin{frame}[fragile]{Serial code}
	Demo...
\end{frame}
%-------------------------------------------------------------------------------

%-------------------------------------------------------------------------------
\begin{frame}[fragile]{Visualization tools}
	%
	\begin{itemize}
		\item For the time being, the simulation is done in two dimensions. This makes it relatively straightforward to paint the particles on the canvas.
		\item In three dimensions, we would need to add behaviour for the third dimension and would need to change the way in which particles are currently being drawn.
		\item To avoid building code for rendering in more than two dimensions, we could rely on visualization tools.
	\end{itemize}
	%
\end{frame}
%-------------------------------------------------------------------------------

%-------------------------------------------------------------------------------
\begin{frame}[fragile]{Visualization tools}
	An option for visualization is using \emph{ParaView} \cite{website:paraview_wiki}.
	%
	\begin{itemize}
		\item Used at the CSCS (Swiss National Supercomputing Centre) \cite{website:cscs}.
		\item Open source, used for visualizing two and three-dimensional data sets.
		\item Platforms supported range from single-processor workstations to multiple-processor distributed-memory supercomputers or workstation clusters.
		\item Many file formats supported.
	\end{itemize}
	%
\end{frame}
%-------------------------------------------------------------------------------

%-------------------------------------------------------------------------------
\begin{frame}[fragile]{Visualization tools}
	An example file format is the following:
	%
	\begin{table}
		\centering
		\begin{tabular}{l l l l}
			\emph{particle index} & \emph{x-coordinate} & \emph{y-coordinate} & \emph{z-coordinate} \\
			0 & 3.5 & 6.0 & 50.0\\
			1 & 1.4 & 3.0 & 10.0\\
			\vdots & \vdots & \vdots & \vdots\\
			1000 & 5.5 & 10.5 & 20.2\\
		\end{tabular}
	\end{table}
	%
\end{frame}
%-------------------------------------------------------------------------------

%-------------------------------------------------------------------------------
\begin{frame}[fragile]{Visualization tools}
	Using the file format [particle index, x-coordinate, y-coordinate], we need to create a file for each time step. In each file, we'll write the position of each particle in the given time step.
	%
	\begin{itemize}
		\item \texttt{positions\_0.txt}
		\item \texttt{positions\_1.txt}
		\item \texttt{positions\_2.txt}
		\item And so on...
	\end{itemize}
	%
\end{frame}
%-------------------------------------------------------------------------------

%-------------------------------------------------------------------------------
\begin{frame}[fragile]{Visualization tools}
	Demo...
\end{frame}
%-------------------------------------------------------------------------------

%-------------------------------------------------------------------------------
\begin{frame}[fragile]{Parallelization methods}
	Preliminary benchmarks made for matrix--matrix multiplication execution time, comparing performance of serial code vs code parallelized using \emph{OpenACC}.
\end{frame}
%-------------------------------------------------------------------------------

%-------------------------------------------------------------------------------
\begin{frame}[fragile]{Parallelization methods}
	% Performance comparison
	\begin{figure}[H]
		\centering
		%
\begin{tikzpicture}

% f(x)
\begin{axis}[
width=0.4\linewidth,
at={(0.0in,0.0in)},
separate axis lines,
every outer x axis line/.append style={black},
every x tick label/.append style={font=\footnotesize},
every outer y axis line/.append style={black},
every y tick label/.append style={font=\footnotesize},
xtick pos=left,
ytick pos=left,
xticklabel style={font=\footnotesize},
yticklabel style={font=\footnotesize},
yminorticks=true,
xmin=1000,
xmax=4000,
ymin=0,
ymax=400,
xlabel={Matrix size $n$},
ylabel={Execution time (seconds)},
label style={font=\footnotesize},
axis background/.style={fill=none},
legend style={at={(0,0)},legend cell align=left,align=left,draw=none,fill=none,font=\footnotesize},
legend entries={Serial,
                OpenACC},
legend pos=north west
]

\addlegendimage{black}
\addlegendimage{mycolor3}

\addplot+[color=black, line width=1.5pt,mark size=1.0pt,mark=none,mark options={fill=black}]
table[x=matrix_size,y=naive]{./Graphics/data/performance_comparison.txt}; \label{plot:avg_time_per_iter}

\addplot+[color=mycolor3, line width=1.5pt,mark size=1.0pt,mark=none,mark options={fill=mycolor3}]
table[x=matrix_size,y=openacc]{./Graphics/data/performance_comparison.txt};
\end{axis}
%

\end{tikzpicture}
%

		\caption{Matrix--matrix multiplication: execution time comparison.}
	\end{figure}
	%
\end{frame}
%-------------------------------------------------------------------------------

%-------------------------------------------------------------------------------
\begin{frame}[fragile]{Project plan}
	Show current project calendar...
\end{frame}
%-------------------------------------------------------------------------------

%-------------------------------------------------------------------------------
\begin{frame}[fragile]{References}
	\printbibliography
\end{frame}
%-------------------------------------------------------------------------------

%-------------------------------------------------------------------------------
\begin{frame}[fragile]{Questions}
	%
	\begin{center}
		Thank you for your attention!\\
		\\
		Any questions?
	\end{center}
	%
\end{frame}
%-------------------------------------------------------------------------------


\end{document}
