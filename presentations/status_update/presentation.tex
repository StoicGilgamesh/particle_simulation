% Dmitry Mikushin, USI Lugano, dmitry.mikushin@usi.ch,
% using portions of original style file by Tom Cashman
%
% IMPORTANT NOTICE:
%
% The USI logo is unique; it is authorized for use only by employees of the
% Università della Svizzera italiana for work-related projects; others can use them
% ONLY with prior authorization (contact: press@usi.ch).
%
% http://www.press.usi.ch/en/corporate-design/corporate-design-stampa.htm
%
% This is an example beamer presentation, which uses Università della Svizzera italiana
% design theme.

\documentclass[aspectratio=169]{beamer}

\usetheme{usi}
\usepackage{multirow}
\usepackage{hhline}
\usepackage{tikz}
\usepackage{graphicx}
\usepackage{caption}
\usepackage{subcaption}
\usepackage{listings}
\usepackage{array}
\usepackage{xcolor}
\usepackage{float}
\usetikzlibrary{shapes.geometric}

\newcommand\score[2]{
\pgfmathsetmacro\pgfxa{#1+1}
\tikzstyle{scorestars}=[star, star points=5, star point ratio=2.25, draw,inner sep=1.3pt,anchor=inner point 3]
  \begin{tikzpicture}[baseline]
    \foreach \i in {1,...,#2} {
    \pgfmathparse{(\i<=#1?"usi@yellow":"gray")}
    \edef\starcolor{\pgfmathresult}
    \draw (\i*2.0ex,0) node[name=star\i,scorestars,fill=\starcolor,color=\starcolor]  {};
   }
  \end{tikzpicture}
}

\definecolor{cadmiumgreen}{rgb}{0.0, 0.42, 0.24}

\setlength{\fboxsep}{0.25pt}%
\setlength{\fboxrule}{0pt}%

\title[Particle Simulations with OpenACC]{\textbf{Particle Simulations with OpenACC: Speedup and Scaling}\\[0.5em] Overview of mathematical models, simulation used, and OpenACC}
\author{Samuel A. Cruz Alegr\'{i}a, Alessandra M. de Felice, Hrishikesh R. Gupta}
\institute{(University of Lugano)}
\date{\today}


\begin{document}
\begin{frame}
\titlepage
\end{frame}
%-------------------------------------------------------------------------------
%-------------------------------------------------------------------------------
\begin{frame}[fragile]{Status Update}

Our tasks for this week were the following:
%
\begin{itemize}
	\item Develop serial code.
	\item Investigate visualization tools.
	\item Investigate parallelization methods.
\end{itemize}
%

\end{frame}
%-------------------------------------------------------------------------------

%-------------------------------------------------------------------------------
\begin{frame}[fragile]{Serial Code}
	The serial code is divided into the following three main sections:
	%
	\begin{enumerate}
		\item Tracing particles (trails or no trails).
		\item Drawing particles.
		\item Updating particle details such as position and velocity.
	\end{enumerate}
	%
	
	For the moment, particle movement doesn't strictly abide to any well-established physics. For instance, particle collisions with each other are not calculated yet.
\end{frame}
%-------------------------------------------------------------------------------

%-------------------------------------------------------------------------------
\begin{frame}[fragile]{Serial Code}
	Demo...
\end{frame}
%-------------------------------------------------------------------------------

%-------------------------------------------------------------------------------
\begin{frame}[fragile]{Visualization Tools}
	%
	\begin{itemize}
		\item For the time being, the simulation is done in two dimensions. This makes it relatively straightforward to paint the particles on the canvas.
		\item In three dimensions, we would need to add behaviour for the third dimension and would need to change the way in which particles are currently being drawn.
		\item In order to minimize time spent in building code for rendering in more than two dimensions, we can choose to use visualization tools.
	\end{itemize}
	%
\end{frame}
%-------------------------------------------------------------------------------

%-------------------------------------------------------------------------------
\begin{frame}[fragile]{Visualization Tools}
	An option for visualization is \emph{ParaView}.
	%
	\begin{itemize}
		\item Used at the CSCS (Swiss National Supercomputing Centre).
		\item Open source, used for visualizing two and three-dimensional data sets.
		\item Platforms supported range from single-processor workstations to multiple-processor distributed-memory supercomputers or workstation clusters.
	\end{itemize}
	%
\end{frame}
%-------------------------------------------------------------------------------

%-------------------------------------------------------------------------------
\begin{frame}[fragile]{References}

\begin{itemize}
\small{\item Farber, R., 2016.Parallel programming with OpenACC. Newnes.
\item Gonzales, R.,  Martin, M., Mittow, N., and Rasmuss, R.,2016, An Introduction to OpenAcc.ECS 158 Final Project.
\item Li, X., Shih, P.C., Overbey, J., Seals, C. and Lim, A., 2016. Comparing programmer productivity in OpenACC and CUDA: an empirical investigation.International Journal of Computer Science, Engineering and Applications (IJCSEA),6(5), pp.1-15.
\item Memeti, S., Li, L., Pllana, S., Kołodziej, J. and Kessler, C., 2017, July. Benchmarking OpenCL, OpenACC, OpenMP, and CUDA: programming productivity, performance, and energy consumption. InProceedings of the 2017 Workshop on Adaptive Resource Management and Scheduling for Cloud Computing(pp. 1-6). ACM.
\item Urbanic, J., 2013. Introduction to Directive Based Programming.
\item OpenACC Programming and Best Practices. Guide \url{http://www.openacc.org/sites/default/files/inline-files/OpenACC_Programming_Guide_0.pdf}
\item \url{http://www.nvidia.com/object/what-is-gpu-computing.html}}
\item Allen, M.P., 2004. Introduction to molecular dynamics simulation. Computational soft matter: from synthetic polymers to proteins, 23, pp.1-28.
\item Eijkhout, V., 2014. Introduction to High Performance Scientific Computing. Lulu. com.
\end{itemize}

\end{frame}
%-------------------------------------------------------------------------------

\end{document}
