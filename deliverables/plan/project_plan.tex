\documentclass[unicode,11pt,a4paper,oneside,numbers=endperiod,openany]{scrartcl}

\usepackage{assignment}
\usepackage{textcomp}
\usepackage{float}


\begin{document}

\setassignment


\serieheader{Software Atelier: Simulation, Data Science \& Supercomputing}{2018}{Authors: Samuel A. Cruz Alegr\'{i}a, Alessandra M. de Felice, Hrishikesh R. Gupta}{}{Project Proposal}{}
\newline

\section{Topic}
\large{\textit{\textbf{Evaluating OpenACC on a large scale particle simulation}}}
\section{Domain}
Physics


\section{Abstract}

\begin{abstract}
The simulation of particle systems has become essential for visualizing the behaviour of relevant physical systems, ranging from simulations of molecular dynamics to simulations of colliding galaxies. Performing realistic simulations require to consider a large number of particles, leading to immense computational costs. Simulating such systems thus require increasingly long time frames and performing increasingly complex simulations may become intractable for single-core simulation tools. Thus, it is essential to develop simulation tools which scale with the number of bodies used in a simulation. A possible approach for scalable simulation tools is to distribute the workload among different parallel  threads available in currently available accelerators. This paper aims to explore the efficiency and scalability of parallelization based on the OpenACC programming standard, which is a directive based standard for parallel computing that offloads the computational kernels to a GPU accelerator.

\section{Goal}

The goal of this project is to optimize particle simulation runtime using OpenACC parallelization to study scaling with an increasing number of particles.

\section{Plan}

The plan for our project is as follows (date format is DD.MM.YY). We note that it may change as we progress:
%
\begin{table}[H]
	\centering
\begin{tabular}{| l | l | l |}
	\hline
	\emph{Task} & \emph{Begin date} & \emph{End date}\\
	\hline
	Study Patrick Zulian's material & 14.03.18 & 18.03.18\\
	\hline
	Develop serial code & 19.03.18 & 28.03.18\\
	\hline
	Work on presentation for week 4/5 & 19.03.18 & 28.03.18\\
	\hline
	Investigate parallelization methods & 19.03.18 & 28.03.18\\
	\hline
	Implement OpenACC in the serial code & 29.03.18 & 18.04.18\\
	\hline
	Gather results \& begin white paper & 19.04.18 & 25.04.18\\
	\hline
	Finalize white paper & 26.04.18 & 21.05.18\\
	\hline
	Prepare final presentation & 22.05.18 & 05.06.18\\
	\hline
	Prepare project poster & 22.05.18 & 15.06.18\\
	\hline
\end{tabular}
\end{table}
%

\section{Deliverables}

The deliverables, based on our project plan, are the following:
%
\begin{itemize}
	\item Project proposal
	\item Project plan (this document)
	\item White paper
	\item Project poster
\end{itemize}
%

Please note that the project proposal has already been delivered.

\section{Milestones}

The milestones, based on our project plan, are the following:
%
\begin{itemize}
	\item Develop serial code
	\item Implement OpenACC in the serial code
	\item Gather results \& begin white paper
	\item Finalize white paper
	\item Project poster
\end{itemize}
%

\section{Contact Details}
%
\begin{tabular}{ l | l }
	Samuel A. Cruz Alegr\'{i}a & samuel.adolfo.cruz.alegria@usi.ch\\
	Alessandra M. de Felice & alessandra.de.felice@usi.ch\\
	Hrishikesh R. Gupta & hrishikesh.gupta@usi.ch
\end{tabular}
%

\end{abstract}

\end{document}
